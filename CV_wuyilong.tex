%-------------------------
% Resume in Latex
% Author : Jake Gutierrez
% Based off of: https://github.com/sb2nov/resume
% License : MIT
%------------------------

\documentclass[letterpaper,11pt]{article}

\usepackage{latexsym}
\usepackage[empty]{fullpage}
\usepackage{titlesec}
\usepackage{marvosym}
\usepackage[usenames,dvipsnames]{color}
\usepackage{verbatim}
\usepackage{enumitem}
\usepackage[hidelinks]{hyperref}
\usepackage{fancyhdr}
\usepackage[english]{babel}
\usepackage{tabularx}
\input{glyphtounicode}


%----------FONT OPTIONS----------
% sans-serif
% \usepackage[sfdefault]{FiraSans}
% \usepackage[sfdefault]{roboto}
% \usepackage[sfdefault]{noto-sans}
% \usepackage[default]{sourcesanspro}

% serif
% \usepackage{CormorantGaramond}
% \usepackage{charter}


\pagestyle{fancy}
\fancyhf{} % clear all header and footer fields
\fancyfoot{}
\renewcommand{\headrulewidth}{0pt}
\renewcommand{\footrulewidth}{0pt}

% Adjust margins
\addtolength{\oddsidemargin}{-0.5in}
\addtolength{\evensidemargin}{-0.5in}
\addtolength{\textwidth}{1in}
\addtolength{\topmargin}{-.5in}
\addtolength{\textheight}{1.0in}

\urlstyle{same}

\raggedbottom
\raggedright
\setlength{\tabcolsep}{0in}

% Sections formatting
\titleformat{\section}{
  \vspace{-4pt}\scshape\raggedright\large
}{}{0em}{}[\color{black}\titlerule \vspace{-5pt}]

% Ensure that generate pdf is machine readable/ATS parsable
\pdfgentounicode=1

%-------------------------
% Custom commands
\newcommand{\resumeItem}[1]{
  \item\small{
    {#1 \vspace{-2pt}}
  }
}

\newcommand{\resumeSubheading}[4]{
  \vspace{-2pt}\item
    \begin{tabular*}{0.97\textwidth}[t]{l@{\extracolsep{\fill}}r}
      \textbf{#1} & #2 \\
      \textit{\small#3} & \textit{\small #4} \\
    \end{tabular*}\vspace{-7pt}
}

\newcommand{\resumeSubSubheading}[2]{
    \item
    \begin{tabular*}{0.97\textwidth}{l@{\extracolsep{\fill}}r}
      \textit{\small#1} & \textit{\small #2} \\
    \end{tabular*}\vspace{-7pt}
}

\newcommand{\resumeProjectHeading}[2]{
    \item
    \begin{tabular*}{0.97\textwidth}{l@{\extracolsep{\fill}}r}
      \small#1 & #2 \\
    \end{tabular*}\vspace{-7pt}
}

\newcommand{\resumeSubItem}[1]{\resumeItem{#1}\vspace{-4pt}}

\renewcommand\labelitemii{$\vcenter{\hbox{\tiny$\bullet$}}$}

\newcommand{\resumeSubHeadingListStart}{\begin{itemize}[leftmargin=0.15in, label={}]}
\newcommand{\resumeSubHeadingListEnd}{\end{itemize}}
\newcommand{\resumeItemListStart}{\begin{itemize}}
\newcommand{\resumeItemListEnd}{\end{itemize}\vspace{-5pt}}

%-------------------------------------------
%%%%%%  RESUME STARTS HERE  %%%%%%%%%%%%%%%%%%%%%%%%%%%%


\begin{document}

%----------HEADING----------
% \begin{tabular*}{\textwidth}{l@{\extracolsep{\fill}}r}
%   \textbf{\href{http://sourabhbajaj.com/}{\Large Sourabh Bajaj}} & Email : \href{mailto:sourabh@sourabhbajaj.com}{sourabh@sourabhbajaj.com}\\
%   \href{http://sourabhbajaj.com/}{http://www.sourabhbajaj.com} & Mobile : +1-123-456-7890 \\
% \end{tabular*}

\begin{center}
    \textbf{\Huge \scshape Yilong Wu} \\ \vspace{1pt}
     \href{mailto:wuyilong2000@outlook.com}{\underline{wuyilong2000@outlook.com}} $|$ 
    \href{https://github.com/Wimacs}{\underline{github.com/Wimacs}}
\end{center}


%-----------EDUCATION-----------
\section{Education}
  \resumeSubHeadingListStart
    \resumeSubheading
      {University of Electronic Science and Technology of China}{Chengdu, Sichuan, China}
      {Bachelor of Engineering in Software Engineering, Elite Program }{Aug. 2018 -- Present}
      \resumeItemListStart
      	\resumeItem{GPA: 3.72/4.00 CET4: 582/710}
        \resumeItem{Excellent course: Computer Architecture(95/100), Compiler(92/100), Operating System(92/100)}
       \resumeItemListEnd:
    \resumeSubheading
      {National University of Singapore}{Singapore }
      {Summer Workshop}{July. 2019 -- Aug 2019}
       \resumeItemListStart
      	\resumeItem{Made a pet feeding robot based on Raspberry Pi}
       \resumeItemListEnd:
  \resumeSubHeadingListEnd




%-----------PROJECTS-----------
\section{Selected Graphics Related Projects}
    \resumeSubHeadingListStart
      \resumeProjectHeading
          {\href{https://github.com/Wimacs/WiRay}{\textbf{WiRay}} $|$ \emph{C++, Intel TBB, Physically based rendering}}{}
          \resumeItemListStart
            \resumeItem{Developed a physically based renderer based on nori}
            \resumeItem{Light Transport Algorithm: PT, BDPT, Photon Mapping, MMLT}
            \resumeItem{Disney BRDF}
            \resumeItem{Build LBVH in parallel on the CPU}
          \resumeItemListEnd
      \resumeProjectHeading
          {\href{https://github.com/Wimacs/WiRay_GPU}{\textbf{WiRay-GPU}} $|$ \emph{C++, CUDA, Physically based rendering}}{}
          \resumeItemListStart
            \resumeItem{Developed a interactable Path Tacer on GPU}
            \resumeItem{Accelerating Data Structure: LBVH, HLBVH, SBVH, TRBVH}
            \resumeItem{Imgui for debuging}
            \resumeItemListEnd
       \resumeProjectHeading
          {\href{https://github.com/Wimacs/taichi_code/tree/master/apf}{\textbf{PIC vs FLIP vs APIC}} $|$ \emph{Taichi,  Python, Physically based animation}}{}
          \resumeItemListStart
            \resumeItem{A hybrid Eulerian–Lagrangian fluid solver}
            \resumeItem{MAC grid finite difference scheme}
            \resumeItem{MGPCG for pressure projection}
            \resumeItem{Bilinear interpolation for P2G and G2P operation}
           \resumeItemListEnd
        \resumeProjectHeading
          {\textbf{\href{https://github.com/Wimacs/taichi_code/tree/master/hw2}{\textbf{PBD}} vs \href{https://github.com/g1n0st/GAMES201/tree/master/hw2}{\textbf{MLS-MPM}} in real-time} $|$ \emph{Taichi,  Python, Physically based animation}}{}
          \resumeItemListStart
            \resumeItem{Final project for GAMES 201 \& CCVR entries}
            \resumeItem{Collision and Stretching constraints in PBD}
            \resumeItem{ Multi-species model for sand-water coupling}
           \resumeItemListEnd
         \resumeProjectHeading
          {\href{https://github.com/Wimacs/taichi_code/tree/master/hw1}{\textbf{Euler Fluid}} $|$ \emph{Taichi,  Python, Physically based animation}}{}
          \resumeItemListStart
            \resumeItem{Jacobi, Gauss-Seidel, CG for pressure projection}
            \resumeItem{Semi-Lagrangian rk1, Semi-Lagrangian rk2, MacCormack, Advection-Reflection for Advection}
            \resumeItem{ Real-time and Interactable}
           \resumeItemListEnd
    \resumeSubHeadingListEnd

%-----------PROGRAMMING SKILLS-----------
\section{Honor and Awards}
  \resumeSubHeadingListStart
    \resumeSubheading
      {\href{https://www.chinavr.info/news/show/id/3659.html}{National First Prize. China Competition on Virtual Reality -  CCVR 2020}}{Jilin, China }
      {A survey about the application of material point method in real-time scenarios }{Aug. 2020}
    \resumeSubheading
      {\href{http://jsjds.ruc.edu.cn/UploadFiles/202082817196435.pdf}{National Second Prize. Chinese undergraduate computer design contest}}{Shandong, China }
      {VR games}{Jun. 2020}
     \resumeSubheading
      {UESTC school-level scholarship}{Sichuan, China }
      {}{Oct. 2019}
 \resumeSubHeadingListEnd

%
%-----------PROGRAMMING SKILLS-----------
\section{Technical Skills}
 \begin{itemize}[leftmargin=0.15in, label={}]
    \small{\item{
     \textbf{Languages}{: C/C++, Python, Taichi, CUDA, C\#, RISC-V ASM, X86 ASM} \\
     \textbf{Frameworks}{: OpenGL, Pytorch, Latex, Unity, Unreal... } \\
     \textbf{Math}{: Calculus, Linear Algebra, Statistics, Probability theory, Numerical Analysis}
    }}
 \end{itemize}


%-------------------------------------------
\end{document}
